\documentclass[12pt]{article}
%
%Preambulo
%%
\usepackage[T1]{fontenc}
\usepackage[utf8]{inputenc}
\usepackage[spanish]{babel}
\usepackage{graphicx}
%%
\parindent = 1cm


\begin{document}
\begin{center}
\includegraphics[width=2.5cm,keepaspectratio]{imagenes/udea_logo.png}
\vspace*{\baselineskip}\\[0.1cm]
{
\bf\fontsize{19}{0}{\selectfont{UNIVERSIDAD DE ANTIOQUIA}\\[0.5cm]
\fontsize{11}{0}{FACULTAD DE INGENIERÍA}\\[0.5cm]
\fontsize{11}{0}{INFORMÁTICA II}\\[0.5cm]
}

\vspace*{\baselineskip}\\[3cm]
\fontsize{11}{0}{ENSAYO: }
\fontsize{11}{0}{¿REALMENTE SON PROBLEMAS?}
\vspace*{\baselineskip}\\[4cm]
\fontsize{11}{0}{JUAN PABLO ZAPATA GÓMEZ}
\vspace*{\baselineskip}\\
\fontsize{11}{0}{C.C. 1.037.977.046}
\end{center}
\newpage

\chapter{
\begin{itemize}
    \item{\LARGE{¿Realmente son Problemas?}}
\end{itemize}

}

Los problemas han existido siempre, su complejidad aumenta a la par de la inteligencia humana, grandes genios en distintas épocas han aportado su grano de arena en el pensamiento humano, su curiosidad y iniciativa en resolver problemas han formado el mundo como se conoce hoy en día, desde satisfacer las necesidades básicas hasta lo más complejos desafíos científicos. La resolución de problemas es clara de visualizar: el bienestar y la comodidad humana dependen de ello, siendo notablemente llamativa y llamando la atención de todas las áreas de estudio en todas sus disciplinas.\\[0.2cm]

La resolución de problemas es la base en las matemáticas, casos ambiguos como: conjuntos compuestos por otros conjuntos o conjuntos en los que el propio conjunto era elemento de sí mismo, eran situaciones contradictorias que algo no estaba bien planteado, algo fallaba. ¿Las matemáticas no eran infalibles? Cuestiones como similares abrieron grandes debates durante el siglo XX llamada la Crisis de fundamentos, personajes como David Hilbert que aseguraban que solo había que solucionar algunos problemas y las matemáticas
volverían a ser como antes, consiste en combinar ideas ya aceptadas y demostrar teoremas nuevos mediante leyes lógicas consistentes, completas y con pasos finitos; este último inconsistente y demostrado por Kurt Gödel. ¿Dejaron de valer las matemáticas? La simple respuesta es no, ahora sabemos que tienen límites, es la semilla de la automatización del pensamiento, la programación y primeros ordenadores, el cual nos encontramos reodeados de ellos.\\[0.2cm]

Fueron muchas personas las que intervinieron en el origen de los ordenadores, un hombre se destacó sobre los demás, Alan Turing y su máquina universal. Los ordenadores son capaces de hacer cosas sumamente increíbles, desde simular encendido en leds hasta
simular sistemas complejos, la informática posee una historia larga, desde máquinas como la de pascal, pasando por la maquina analítica de Babbage hasta los más modernos y costosos Smartphone, en su inicio eran artefactos mecánicos enfocadas en el cálculo, ya pasaron a maquinas programables y el ordenador moderno; en la memoria no solo se guardan datos, si no los propios programas que se ejecutan, pero hablar de memoria ya es otro tema, no tan moderno y muy interesante. En su aporte a las matemáticas, lógica y criptografía resolvió el gran dilema que había propuesto Hilbert poniendo las bases de la teoría de la computación. La máquina universal es el modelo en el que asientan todos los ordenadores actuales pensada por Turing incluso antes de existir el primer ordenador y de desarrollar los inicios de la teoría computacional, pensando incluso en la IA.\\[0.2cm]

“Turing concibe máquinas con la capacidad de sentir, lo que visto desde hoy día lo sitúa como promotor de una inteligencia artificial fuerte; sin embargo, su propuesta matemática y computacional concibe al pensamiento como cálculo de tareas matemáticas, lo que sugiere que aboga más bien por una inteligencia artificial débil”.\cite{Aliseda}\\[0.2cm]

El resultado de dilemas matemáticos y la intervención de genios en el paso del tiempo, construyeron de manera de pensamientos, lógica y mecánica, las bases y nacimientos de la computación, al día de hoy se disfrita de grandes avances tecnológicos e incluso es loco pensar en un mundo donde no exista la tecnología, una civilización cada vez mas competitia y globalizada enfocada en la implementación de estrategias en diversos sectores donde se desempeña el ser humano.\\[0.2cm]

Así finalizamos en lo que anteriormente era quizá como una piedra en un zapato, pasó a ser en inicio y un gran paso a la tecnlogía y comunicación moderna, siendo así parte fundamental en nuestro entorno, provocanso aspi temas de aislamiento e incluso discriminación en ámbitos culturales y sociales, pero aún así el aporte de estos avances, técnicamente hacen parte de nuestro crecimiento como civilización.\\[0.2cm]

\newpage
\begin{thebibliography}{1}
\bibitem{Aliseda}Aliseda, ¿Inteligencia mecánica? Academia Mexicana de Ciencias, Páginas 8. https://biblat.unam.mx/es/revista/ciencia-academia-mexicana-de-ciencias/articulo/inteligencia-mecanica-la-pregunta-de-alan-turing

\bibitem{Lahoz-Beltra}Lahoz-Beltra, (08 de 2013). Alan Turing y los orígenes de la investigación multidisciplinar. Recuperado el 2013, de Universidad Autónoma de Madrid. Fundación General: https://repositorio.uam.es/bitstream/handle/10486/678743/EM444.pdf?\\sequence=1isAllowed=y

\bibitem{History Channel}History Channel,(año desconocido de publicación). La historia de las matemáticas - 4. Hacia el Infinito y más allá. https://www.youtube.com/watch?v=87FAhhedvU : joven matemático Aleman, Hilbert.

\end{thebibliography}

\end{document}
